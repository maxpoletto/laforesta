Come già anticipato, gli interventi nei cedui vengono effettuati a turni di 12, 15, o 18 anni, a seconda delle condizioni del sopprassuolo.

\begin{description}
    \item[Turno di 12 anni: avviamento a fustaia.] Nei cedui di castagno dove è presente la rinnovazione di conifere, si interviene a intervalli di 12 anni con rilascio di 50 matricine/ha per due turni, in modo da favorire la competizione verticale e l'avviamento a fustaia.

    \item[Turno di 15 anni: gestione ordinaria.] Dove le condizioni stazionali sono normali, si interviene a intervalli di 15 anni con rilascio di 50 matricine/ha per due turni, in attesa dell'insediamento della rinnovazione di conifere o di interventi di semina.

    \item[Turno di 18 anni: miglioramento stazionale.] Dove il bosco è più rado, si interviene a intervalli di 18 anni con rilascio di 75 matricine/ha per due turni, con l'obbiettivo di incrementare la copertura e la lettiera e migliorare la fertilità stazionale.
\end{description}

Di seguito il calendario degli interventi pianificati per i cedui:

