Di seguito il risassunto dei volumi totali di ciascuna compresa, e poi le stime suddivise per singole particelle. Come descritto nella sezione~\ref{sec:note-volumi}, questi volumi, a differenza di quelli usati per il calcolo della ripresa, includono anche le piante al di sotto della classe diametrica 20 cm. I valori inferiori e superiori sono relativi all'intervallo fiduciario al 95\%.

@@tsv(alberi=alberi-calcolati.csv,per_compresa=si,per_particella=no,per_genere=no,intervallo_fiduciario=si,totali=si)

\subsection*{Stima dei volumi: Serra}
@@tsv(alberi=alberi-calcolati.csv,compresa=Serra,per_compresa=no,per_particella=si,per_genere=no,intervallo_fiduciario=si,totali=si)
\clearpage

\subsection*{Stima dei volumi: Fabrizia}
@@tsv(alberi=alberi-calcolati.csv,compresa=Fabrizia,per_compresa=no,per_particella=si,per_genere=no,intervallo_fiduciario=si,totali=si)

\subsection*{Stima dei volumi: Capistrano}
@@tsv(alberi=alberi-calcolati.csv,compresa=Capistrano,per_compresa=no,per_particella=si,per_genere=no,intervallo_fiduciario=si,totali=si)

\clearpage
