\section{Ripresa attuale}
\label{sec:ripresa}

La ripresa per le particelle governate a fustaia è calcolata secondo i criteri definiti dalle Prescrizioni di Massima e di Polizia Forestale della Regione Calabria del 2011, e in particolare dagli articoli 48 e 49.

In sintesi, la provvigione minima di una particella dipende dalla provvigione minima del comparto corrispondente, secondo la seguente tabella:

\begin{center}
\begin{tabular}{llr}
\hline
         &                 & Provvigione \\
Comparto & Caratteristiche & minima (m³/ha)\\
\hline
A        & Popolamenti a prevalenza di faggio, o misti di conifere e latifoglie & 350 \\
B        & Popolamenti a prevalenza di faggio, o misti di conifere e latifoglie & 350 \\
C        & Pinete di pino nero e laricio, popolamenti misti & 250 \\
D        & Pinete di pino nero e laricio, popolamenti misti & 250 \\
E        & Popolamenti a prevalenza di faggio, o misti di conifere e latifoglie & 350 \\
\hline
\end{tabular}
\end{center}

Il prelievo massimo consentito in una particella è calcolato in base a due criteri, l'entità della provvigione e l'età media del popolamento.

Il volume massimo del prelievo in una particella è determinato dall'eccesso della provvigione effettiva rispetto alla provvigione minima della particella, secondo la seguente tabella.

\begin{center}
\begin{tabular}{lr}
\hline
Eccedenza percentuale sulla         & Prelievo massimo \\
provvigione minima                  & percentuale \\
\hline
$\ge 80$        & 25 \\
$\ge 60, < 80$  & 20 \\
$\ge 40, < 60$  & 15 \\
$\ge 20, < 40$  & 10 \\
$< 20$          &  0 \\
\hline
\end{tabular}
\end{center}

Inoltre, per fustaie fino a 60 anni di età, il prelievo non può superare una determinata percentuale dell'area basimetrica, secondo la seguente tabella:
\begin{center}
\begin{tabular}{lr}
\hline
Età media & Percentuale massima \\
(anni)    & area basimetrica \\
\hline
$\ge 30, < 60$  & 20 \\
$<30$           & 15 \\
\hline
\end{tabular}
\end{center}

In base a queste regole, la ripresa totale per compresa è la seguente.

@@tpt(alberi=alberi-calcolati.csv,per_compresa=si,per_particella=no,col_prelievo=si,totali=si)

Di seguito, per ciascuna compresa, la ripresa per particella. Per ogni particella riportiamo anche il comparto e l'età media per permettere di verificare manualmente le percentuali massime di prelievo secondo i criteri di cui sopra.

\subsection*{Ripresa: Serra}
\begingroup
    \setlength{\LTleft}{0pt}
    {\footnotesize
@@tpt(alberi=alberi-calcolati.csv,compresa=Serra,per_compresa=no,totali=si)
    }
\endgroup

\subsection*{Ripresa: Fabrizia}
\begingroup
    \setlength{\LTleft}{0pt}
    {\footnotesize
@@tpt(alberi=alberi-calcolati.csv,compresa=Fabrizia,per_compresa=no,totali=si)
    }
\endgroup

\subsection*{Ripresa: Capistrano}
\begingroup
    \setlength{\LTleft}{0pt}
    {\footnotesize
@@tpt(alberi=alberi-calcolati.csv,compresa=Capistrano,per_compresa=no,totali=si)
    }
\endgroup

\clearpage
