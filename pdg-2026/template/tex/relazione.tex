\documentclass[a4paper,11pt]{article}
\usepackage[utf8]{inputenc}
\usepackage[italian]{babel}
\usepackage{url}
\usepackage{graphicx}
\usepackage{geometry}
\usepackage{hyperref}
\usepackage{float}
\usepackage{longtable}
\usepackage{verbatim}
\geometry{margin=2cm}
\setlength{\LTleft}{1cm}

% Typography: serif body text
\usepackage{libertinus}
\usepackage[T1]{fontenc}
% Sans-serif section titles (Gill Sans-like)
\usepackage{gillius}
\usepackage{sectsty}
\allsectionsfont{\sffamily}

\usepackage[style=authoryear]{biblatex}
\addbibresource{ref.bib}

\title{\sffamily Documentazione integrativa al Piano di Assestamento}
\date{\sffamily Febbraio 2026}
\author{\sffamily Società Agricola La Foresta\\\sffamily Serra San Bruno}

\begin{document}
\maketitle

\section{Introduzione}

Il presente documento contiene la documentazione integrativa al Piano di Assestamento della Società Agricola La Foresta, come richiesto dal Dipartimento Politiche della Montagna, Foreste, Forestazione e Difesa del Suolo della Regione Calabria con Protocollo 464707 del 25/06/2025.

La relazione si articola in 7 parti.
\begin{itemize}
\item La parte~\ref{sec:note-metodologia} chiarisce alcuni metodi usati per redigere il piano, in particolare per quanto riguarda il calcolo degli incrementi e l'elaborazione del programma di taglio.
\item La parte~\ref{sec:ipsometrie-diametri} risponde al punto 1 della richiesta della regione: illustra per ogni specie principale le relative curve ipsometriche con equazioni interpolanti e la distribuzione grafica in classi diametriche del numero di alberi per ettaro.
\item La parte~\ref{sec:volumi} risponde al punto 2 della richiesta: presenta la cubatura di tutti i soprassuoli usando le tavole prescritte dalle linee guida.
\item La parte~\ref{sec:ripresa} risponde al punto 3 della richiesta: quantifica la ripresa in relazione alla stima dei volumi.
\item La parte~\ref{sec:ceduo} risponde al punto 4 della richiesta: espone il piano di regolarizzazione dei cedui di castagno.
\item La parte~\ref{sec:piano} presenta un piano di tagli aggiornato, redatto tenendo conto degli incrementi secondo i criteri descritti nella parte~\ref{sec:note-metodologia}.
\item La parte~\ref{sec:per-particella}, infine, risponde ai punti 5 e 6 della richiesta. Per ogni particella, indica le necessità colturali dei soprassuoli, descrive gli interventi colturali da eseguire, e fornisce, per ogni specie e classe diametrica, il numero di alberi per ettaro, il volume e l'area basimetrica per ettaro, la distribuzione delle altezze, e gli incrementi percentuali e correnti. Espone inoltre i volumi e la ripresa totali, suddivisi per specie.
\end{itemize}

\section{Note metodologiche}
\label{sec:note-metodologia}

Di seguito sono riportati alcuni chiarimenti sui rilevamenti e i metodi di calcolo usati nella stesura di questa relazione.

\subsection{Campionamento}

Le 177 aree di saggio già descritte nella relazione iniziale, ciascuna della dimensione di $\frac{1}{8}$ ha, sono state soggette a cavallettamento totale, fornendo i diametri di circa 9,900 piante. Per un sottoinsieme casuale di 1880 piante è stata rilevata anche l'altezza tramite ipsometro laser. Per un ulteriore sottoinsieme di circa 200 piante è stato misurato l'incremento periodico di raggio $L_{10}$, ovvero lo spessore dei 10 anelli legnosi periferici successivi.

\subsection{Calcolo dei volumi}

I volumi dei soprassuoli sono calcolati secondo la metodologia alle pp.20--27 di~\cite{tabacchi:2011}.

Oltre alla cubatura totale, riportiamo una stima del numero di piante, nonché l'intervallo fiduciario al 95\% della stima dei volumi, ottenuto sempre con la metodologia del Tabacchi. Il margine d'errore per il volume totale è calcolato come la somma dei margini d'errore di ciascuna specie: rappresenta quindi il caso peggiore, in cui le particelle sono completamente correlate tra loro.

I valori elencati includono le piante sottomisura, al di sotto dei 17,5 cm di diametro (classe diametrica inferiore a 20 cm). Per contro, naturalmente, i volumi usati per la ripresa escludono le piante sottomisura.


\subsection{Calcolo della ripresa}

\subsection{Calcolo degli incrementi}

\subsection{Aggiornamento del programma dei tagli}

\input{sec-ipsometrie-diametri}
Di seguito il risassunto dei volumi totali di ciascuna compresa, e poi le stime suddivise per singole particelle. I valori inferiori e superiori sono relativi all'intervallo fiduciario al 95\%.

@@tsv(alberi=alberi-calcolati.csv,per_compresa=si,per_particella=no,per_genere=no,intervallo_fiduciario=si,totali=si)

\subsection*{Stima dei volumi: Serra}
@@tsv(alberi=alberi-calcolati.csv,compresa=Serra,per_compresa=no,per_particella=si,per_genere=no,intervallo_fiduciario=si,totali=si)
\clearpage

\subsection*{Stima dei volumi: Fabrizia}
@@tsv(alberi=alberi-calcolati.csv,compresa=Fabrizia,per_compresa=no,per_particella=si,per_genere=no,intervallo_fiduciario=si,totali=si)

\subsection*{Stima dei volumi: Capistrano}
@@tsv(alberi=alberi-calcolati.csv,compresa=Capistrano,per_compresa=no,per_particella=si,per_genere=no,intervallo_fiduciario=si,totali=si)

\clearpage

In base alle regole descritte nella sezione~\ref{sec:note-ripresa}, la ripresa totale per compresa è la seguente.

@@tpt(alberi=alberi-calcolati.csv,per_compresa=si,per_particella=no,col_prelievo=si,totali=si)

Di seguito, per ciascuna compresa, la ripresa per particella. Per ogni particella riportiamo anche il comparto e l'età media per permettere di verificare manualmente le percentuali massime di prelievo secondo i criteri di cui sopra.

\subsection*{Ripresa: Serra}
@@tpt(alberi=alberi-calcolati.csv,compresa=Serra,per_compresa=no,totali=si)

\subsection*{Ripresa: Fabrizia}
@@tpt(alberi=alberi-calcolati.csv,compresa=Fabrizia,per_compresa=no,totali=si)

\subsection*{Ripresa: Capistrano}
@@tpt(alberi=alberi-calcolati.csv,compresa=Capistrano,per_compresa=no,totali=si)

\clearpage

Come già anticipato, gli interventi nei cedui vengono effettuati a turni di 12, 15, o 18 anni, a seconda delle condizioni del sopprassuolo.

\begin{description}
    \item[Turno di 12 anni: avviamento a fustaia.] Nei cedui di castagno dove è presente la rinnovazione di conifere, si interviene a intervalli di 12 anni con rilascio di 50 matricine/ha per due turni, in modo da favorire la competizione verticale e l'avviamento a fustaia.

    \item[Turno di 15 anni: gestione ordinaria.] Dove le condizioni stazionali sono normali, si interviene a intervalli di 15 anni con rilascio di 50 matricine/ha per due turni, in attesa dell'insediamento della rinnovazione di conifere o di interventi di semina.

    \item[Turno di 18 anni: miglioramento stazionale.] Dove il bosco è più rado, si interviene a intervalli di 18 anni con rilascio di 75 matricine/ha per due turni, con l'obbiettivo di incrementare la copertura e la lettiera e migliorare la fertilità stazionale.
\end{description}

Di seguito il calendario degli interventi pianificati per i cedui:


@@tpdt(alberi=alberi-calcolati.csv,calendario=calendario-tagli-2011-2025.csv,volume_obiettivo=20000,totali=si)

\section{Dettaglio per particella}
\label{sec:per-particella}

Di seguito, per ogni compresa e ogni particella, riportiamo in dettaglio le caratteristiche della particella e dei soprassuoli, il piano degli interventi silvicolturali, e i volumi e la ripresa divisi per specie.

\setlength{\LTleft}{0pt}
@@particelle(compresa=Serra,modello=particella)
@@particelle(compresa=Fabrizia,modello=particella)
@@particelle(compresa=Capistrano,modello=particella)


\printbibliography

\end{document}
