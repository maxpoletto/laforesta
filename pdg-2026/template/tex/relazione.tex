\documentclass[a4paper,11pt]{article}
\usepackage[utf8]{inputenc}
\usepackage[italian]{babel}
\usepackage{url}
\usepackage{graphicx}
\usepackage{geometry}
\usepackage{hyperref}
\usepackage{float}
\usepackage{longtable}
\usepackage{verbatim}
\geometry{margin=2cm}
\setlength{\LTleft}{1cm}

% Typography: serif body text
\usepackage{libertinus}
\usepackage[T1]{fontenc}
% Sans-serif section titles (Gill Sans-like)
\usepackage{gillius}
\usepackage{sectsty}
\allsectionsfont{\sffamily}

\usepackage[style=authoryear]{biblatex}
\addbibresource{ref.bib}

\newif\ifincludesections \includesectionsfalse
\newcommand{\maybeinput}[1]{\ifincludesections \input{#1} \fi}

\title{\sffamily Documentazione integrativa al Piano di Assestamento}
\date{\sffamily Febbraio 2026}
\author{\sffamily Società Agricola La Foresta\\\sffamily Serra San Bruno}

\begin{document}
\maketitle

\section{Introduziones}

Il presente documento contiene la documentazione integrativa al Piano di Assestamento della Società Agricola La Foresta, come richiesto dal Dipartimento Politiche della Montagna, Foreste, Forestazione e Difesa del Suolo della Regione Calabria con Protocollo 464707 del 25/06/2025.

La relazione si articola in 7 parti.
\begin{itemize}
\item La parte~\ref{sec:note-metodologia} chiarisce alcuni metodi usati per redigere il piano, in particolare per quanto riguarda il calcolo degli incrementi e l'elaborazione del programma di taglio.
\item La parte~\ref{sec:ipsometrie-diametri} risponde al punto 1 della richiesta della regione: illustra per ogni specie principale le relative curve ipsometriche con equazioni interpolanti e la distribuzione grafica in classi diametriche del numero di alberi per ettaro.
\item La parte~\ref{sec:volumi} risponde al punto 2 della richiesta: presenta la cubatura di tutti i soprassuoli usando le tavole prescritte dalle linee guida.
\item La parte~\ref{sec:ripresa} risponde al punto 3 della richiesta: quantifica la ripresa in relazione alla stima dei volumi.
\item La parte~\ref{sec:ceduo} risponde al punto 4 della richiesta: espone il piano di regolarizzazione dei cedui di castagno.
\item La parte~\ref{sec:piano} presenta un piano di tagli aggiornato, redatto tenendo conto degli incrementi secondo i criteri descritti nella parte~\ref{sec:note-metodologia}.
\item La parte~\ref{sec:per-particella}, infine, risponde ai punti 5 e 6 della richiesta. Per ogni particella, indica le necessità colturali dei soprassuoli, descrive gli interventi colturali da eseguire, e fornisce, per ogni specie e classe diametrica, il numero di alberi per ettaro, il volume e l'area basimetrica per ettaro, la distribuzione delle altezze, e gli incrementi percentuali e correnti. Espone inoltre i volumi e la ripresa totali, suddivisi per specie.
\end{itemize}

\section{Note metodologiche}
\label{sec:note-metodologia}

Di seguito sono riportati alcuni chiarimenti sui rilevamenti e i metodi di calcolo usati nella stesura di questa relazione.

\subsection{Campionamento}

Le 177 aree di saggio già descritte nella relazione iniziale, ciascuna della dimensione di $\frac{1}{8}$ ha, sono state soggette a cavallettamento totale, fornendo i diametri di circa 9,900 piante. Per un sottoinsieme casuale di 1880 piante è stata rilevata anche l'altezza tramite ipsometro laser. Per un ulteriore sottoinsieme di circa 200 piante è stato misurato l'incremento periodico di raggio $L_{10}$, ovvero lo spessore dei 10 anelli legnosi periferici successivi.

\subsection{Calcolo delle curve ipsometriche}
\label{sec:curve-ipso}
Le curve ipsometriche sono state calcolate per ogni combinazione di specie e compresa comprendente più di 10 alberi campione. L'equazione interpolante ha forma logaritmica, $h = a \ln d + b$, dove $d$ è il diametro in cm e $h$ è l'altezza in metri.

\subsection{Calcolo dei volumi}

I volumi dei soprassuoli sono calcolati secondo la metodologia alle pp.20--27 di~\cite{tabacchi:2011}. Nello specifico:

\begin{itemize}
    \item Per ogni albero cavallettato, l'altezza è stimata usando le equazioni interpolanti di cui sopra.
    \item Per tutti gli alberi di una singola specie in ciascuna area di saggio, il volume del fusto e dei rami grossi viene calcolato usando un'equazione di previsione della forma $v = b_1 + b_2d^2 + b_3d$, dove i coefficienti $b_i$ provengono da~\cite{tabacchi:2011}. Analogamente, la varianza della somma di questi volumi è calcolata usando le matrici di varianza e i coefficienti del Tabacchi.
    \item L'intervallo fiduciario della stima dei volumi dell'intera area di saggio è dato da $\sqrt{\sum_{s}{f_s^2}}$, dove $f_s$ è l'intervallo fiduciario di ciascuna specie, perché le equazioni di previsione sono state calibrate sulla base di campioni indipendenti.
    \item Infine, il volume dell'intera particella viene calcolato scalando la somma delle superfici delle aree di saggio all'intera particella. Anche in questo caso l'intervallo fiduciario è prudenziale, e presume che tutte le superfici di una particella siano completamente correlate tra di loro.
\end{itemize}

Da notare che i valori elencati nella sezione~\ref{sec:volumi} includono le piante sottomisura, al di sotto dei 17,5 cm di diametro (classe diametrica inferiore a 20 cm). Per contro, naturalmente, i volumi usati per la ripresa escludono queste piante.

\subsection{Calcolo della ripresa}

\cite{corona:2007}
\cite{marziliano:2011}
\cite{tabacchi:2011}
\cite{usher:1966}

\subsection{Calcolo degli incrementi}

\subsection{Aggiornamento del programma dei tagli}

\section{Curve ipsometriche e classi diametriche}
\label{sec:ipsometrie-diametri}
\maybeinput{sec-ipsometrie-diametri}

\section{Stima dei volumi}
\label{sec:volumi}
\maybeinput{sec-volumi}

\section{Ripresa attuale}
\label{sec:ripresa}
\maybeinput{sec-ripresa}

\section{Ceduo}
\label{sec:ceduo}
\maybeinput{sec-ceduo}

\section{Piano dei tagli}
\label{sec:piano}
\maybeinput{sec-piano}

\section{Dettaglio per particella}
\label{sec:per-particella}
\maybeinput{sec-per-particella}

\printbibliography

\end{document}
