\documentclass[a4paper,11pt]{article}
\usepackage[utf8]{inputenc}
\usepackage[italian]{babel}
\usepackage{amsmath}
\usepackage{cancel}
\usepackage{url}
\usepackage{graphicx}
\usepackage{geometry}
\usepackage{hyperref}
\usepackage{float}
\usepackage{longtable}
\usepackage{verbatim}
\geometry{margin=2cm}
\setlength{\LTleft}{1cm}

% Typography: serif body text
\usepackage{libertinus}
\usepackage[T1]{fontenc}
% Sans-serif section titles (Gill Sans-like)
\usepackage{gillius}
\usepackage{sectsty}
\allsectionsfont{\sffamily}
\setlength{\parindent}{0pt}
\usepackage[style=alphabetic,backend=biber]{biblatex}
\addbibresource{ref.bib}

\newif\ifincludesections \includesectionsfalse
\newcommand{\maybeinput}[1]{\ifincludesections \input{#1} \fi}
\newcommand{\nota}[1]{{\bf {\em #1}}}

\title{\sffamily Documentazione integrativa al Piano di Assestamento}
\date{\sffamily Febbraio 2026}
\author{\sffamily Società Agricola La Foresta\\\sffamily Serra San Bruno}

\begin{document}
\maketitle

\section{Introduzione}

Il presente documento contiene la documentazione integrativa al Piano di Assestamento della Società Agricola La Foresta, come richiesto dal Dipartimento Politiche della Montagna, Foreste, Forestazione e Difesa del Suolo della Regione Calabria con Protocollo 464707 del 25/06/2025.

La relazione si articola in 7 parti.
\begin{itemize}
\item La parte~\ref{sec:note-metodologia} chiarisce alcuni metodi usati per redigere il piano, in particolare per quanto riguarda il calcolo degli incrementi e l'elaborazione del programma di taglio.
\item La parte~\ref{sec:ipsometrie} risponde al punto 1 della richiesta della regione: illustra per ogni specie principale le relative curve ipsometriche con equazioni interpolanti e la distribuzione grafica in classi diametriche del numero di alberi per ettaro.
\item La parte~\ref{sec:volumi} risponde al punto 2 della richiesta: presenta la cubatura di tutti i soprassuoli usando le tavole prescritte dalle linee guida.
\item La parte~\ref{sec:ripresa} risponde al punto 3 della richiesta: quantifica la ripresa in relazione alla stima dei volumi.
\item La parte~\ref{sec:ceduo} risponde al punto 4 della richiesta: espone il piano di regolarizzazione dei cedui di castagno.
\item La parte~\ref{sec:piano} presenta un piano di tagli aggiornato, redatto tenendo conto degli incrementi secondo i criteri descritti nella parte~\ref{sec:note-metodologia}.
\item La parte~\ref{sec:particelle}, infine, risponde ai punti 5 e 6 della richiesta. Per ogni particella, indica le necessità colturali dei soprassuoli, descrive gli interventi colturali da eseguire, e fornisce, per ogni specie e classe diametrica, il numero di alberi per ettaro, il volume e l'area basimetrica per ettaro, la distribuzione delle altezze, e gli incrementi percentuali e correnti. Espone inoltre i volumi e la ripresa totali, suddivisi per specie.
\end{itemize}

\section{Note metodologiche}
\label{sec:note-metodologia}

Di seguito sono riportati alcuni chiarimenti sui rilevamenti e i metodi di calcolo usati nella stesura di questa relazione.

\subsection{Campionamento}

Le 177 aree di saggio già descritte nella relazione iniziale, ciascuna della dimensione di $\frac{1}{8}$ ha, sono state soggette a cavallettamento totale, fornendo i diametri di circa 9,900 piante. Per un sottoinsieme casuale di 1880 piante è stata rilevata anche l'altezza tramite ipsometro laser. Per un ulteriore sottoinsieme di \nota{circa 200 piante} è stato misurato l'incremento periodico di raggio $L_{10}$, ovvero lo spessore dei 10 anelli legnosi periferici successivi.

\subsection{Calcolo delle curve ipsometriche}
\label{sec:note:ipsometrie}
Le curve ipsometriche sono state calcolate per regressione sulle altezze e i diametri delle 1880 piante misurate, per ogni combinazione di specie e compresa comprendente più di 10 alberi campione. Le equazioni interpolante hanno forma logaritmica, $h = a \ln d + b$, dove $d$ è il diametro in cm e $h$ è l'altezza in metri.

\subsection{Calcolo dei volumi}
\label{sec:note-volumi}

I volumi presenti in ciascuna particella $p$ sono stimati secondo la metodologia alle pp.20--27 di Tabacchi~\cite{tabacchi:2011}. Nello specifico:

\begin{enumerate}
    \item Per ogni albero cavallettato, l'altezza è stimata usando le equazioni interpolanti di cui sopra.
    \item Per tutti gli alberi di una singola specie $f$ campionati in $p$, il volume del fusto e dei rami grossi viene calcolato usando un'equazione di previsione della forma $v_f = b_{f0} + b_{f1}d^2h + b_{f2}d$, dove i coefficienti $b_{fi}$ provengono dal Tabacchi. Analogamente, il margine di errore $e_f$ della stima di questo volume è calcolato usando le matrici di varianza e i coefficienti del Tabacchi.
    \item Il volume totale degli alberi campionati è dato quindi da $v_c = \sum_{f} v_f$. Il margine di errore è calcolato prudenzialmente come $e_c = \sum_{f} e_f$: ciò presume che le piante delle diverse specie nella particella siano perfettamente correlate, nonostante le equazioni di previsione di Tabacchi siano state calibrate sulla base di campioni indipendenti.
    \item Calcoliamo quindi un fattore di scala $s = a / a_s$, dove~$a$ è la superficie di~$p$ e~$a_s$ è la superficie complessiva delle aree di saggio in~$p$.
    \item Il volume dell'intera particella viene calcolato scalando il volume stimato nelle aree di saggio, $v = s\cdot v_c$, e il margine di errore $e = s\cdot e_c$. Pure questo margine di errore è prudenziale, nel senso che presume che tutte le superfici di una particella siano correlate tra di loro. I limiti deli intervalli fiduciari al 95\% elencati nella sezione~\ref{sec:volumi} sono $v \pm e$.
\end{enumerate}

Da notare che i valori elencati nella sezione~\ref{sec:volumi} includono le piante sottomisura, al di sotto dei 17,5 cm di diametro (classe diametrica inferiore a 20 cm). Per contro, naturalmente, i volumi usati per il calcolo della ripresa e degli interventi di taglio escludono queste piante.

\subsection{Calcolo della ripresa}
\label{sec:note-ripresa}

Secondo le prescrizioni di massima della Regione Calabria~\cite{calabria:2011}, artt. 48 e 49, il prelievo durante i tagli colturali nelle fustaie è limitato in funzione della massa disponibile e dell'area basimetrica di ciascuna particella.

In base alla classe economica del comparto, si stabilisce una provvigione minima. Per la nostra azienda, le classi economiche sono le seguenti:

\begin{center}
    \small
    \begin{tabular}{llr}
    \hline
    Classe           &                                          & Provvigione \\
    economica        & Caratteristiche                          & minima (m³/ha)\\
    \hline
    A                & Fustaia di abete                         & 350 \\
    B                & Fustaia mista di conifere e latifoglie   & 350 \\
    C                & Fustaia di pino                          & 250 \\
    D                & Fustaia di pino nero e latifoglie        & 250 \\
    E                & Fustaia di faggio e conifere             & 350 \\
    \hline
    \end{tabular}
\end{center}

Il prelievo massimo volumetrico è quindi dato dalla seguente tabella:

\begin{center}
    \small
    \begin{tabular}{lr}
    \hline
    Eccedenza su provvigione minima         & Prelievo massimo \\
    \hline
    $\ge 80\%$                              & 25\% \\
    $\ge 60\%, < 80\%$                      & 20\% \\
    $\ge 40\%, < 60\%$                      & 15\% \\
    $\ge 20\%, < 40\%$                      & 10\% \\
    $< 20\%$                                &  0\% \\
    \hline
    \end{tabular}
\end{center}

Inoltre, per fustaie fino a 60 anni di età, il prelievo non può superare una determinata percentuale dell'area basimetrica, secondo la seguente tabella:

\begin{center}
    \small
    \begin{tabular}{lr}
    \hline
    Età media                   & Limite area basimetrica \\
    \hline
    $\ge 30$ anni, $< 60$ anni  & 20\% \\
    $<30$ anni                  & 15\% \\
    \hline
    \end{tabular}
\end{center}

Per ottemperare a questi duplici requisiti, procediamo con una simulazione albero per albero basata sulle piante campionate. Per ogni particella~$p$ di area $a$~(ha):

\begin{enumerate}
\item Calcoliamo il volume totale $v$ (m$^3$) delle piante mature (vedere sezione~\ref{sec:note-volumi}) e l'area basimetrica totale $g$ (m$^2$) delle stesse.

\item In base alla classe economica di $p$ e alla provvigione $v/a$, ricaviamo dalle prime due tabelle sopra il prelievo massimo percentuale $pp_{\max}$, e poniamo il prelievo massimo assoluto $V_{\max} = v \cdot pp_{\max}/100$.

\item In base all'età media di $p$, ricaviamo dalla terza tabella il limite percentuale di area basimetrica $gp_{\max}$, e poniamo il limite assoluto $G_{\max} = g \cdot gp_{\max}/100$.

\item Ordiniamo le piante mature campionate per classe diametrica crescente (dai 20~cm in su), in modo da simulare un taglio colturale dal basso volto a rafforzare le piante di maggiori dimensioni.

\item Scorriamo le piante nell'ordine stabilito, accumulando il volume e l'area basimetrica (scalati per $s$, come sopra nella sezione~\ref{sec:note-volumi}) di ciascun albero $b$. Per ogni albero $b$, se l'aggiunta di $b$ non farebbe superare né $V_{\max}$ né $G_{\max}$, lo includiamo nel prelievo; altrimenti la simulazione si arresta. Il volume accumulato al termine costituisce il prelievo massimo per la particella.
\end{enumerate}

\subsection{Calcolo degli incrementi}

Per un sottoinsieme di alberi campione misuriamo l'incremento periodico di raggio $L_{10}$ (spessore degli ultimi 10 anelli legnosi, in mm) tramite carotaggio o abbattimento.\footnote{\nota{Nel presente piano l'incremento è stato stimato
attribuendo a ogni albero campione il $L_{10}$ di un albero carotato nella
stessa area di saggio, indipendentemente dalla classe diametrica.}} Calcoliamo l'incremento percentuale di volume dell'$i$-esimo albero, $pv_i$, secondo la formula di Pressler. Definendo l'incremento corrente di diametro in cm anno$^{-1}$

\[ \Delta d_i = \frac{2L_{10}}{100} \]

otteniamo

\[ pv_i = c \cdot \frac{100\Delta d_i}{d_i} = c \cdot \frac{2L_{10}}{d_i} \]

dove $d_i$ è il diametro (cm) e $c$ è il coefficiente di Pressler. Prudenzialmente poniamo $c = 2$ per tutta la superficie boscata, nonostante studi svolti sull'Appennino Calabrese suggeriscano valori più alti (\cite{marziliano:2011} riscontra coefficienti di Schneider $K$ nell'intervallo $415$--$715$; analiticamente $c = K / 200$).

Per ogni area assestamentale (particella o compresa) e specie, determiniamo l'incremento percentuale medio $pv_j$ della classe diametrica $j$ quale media ponderata sui volumi degli alberi carotati appartenenti a quella classe, come in~\cite{corona:2007}:

\[ pv_j = \frac{\sum_{i=1}^{m_j} pv_{ij}v_{ij}}{\sum_{i=1}^{m_j} v_{ij}} \]

dove $pv_{ij}$ è l'incremento percentuale di volume dell'$i$-esimo albero carotato della classe diametrica $j$, $v_{ij}$ è il suo volume calcolato come sopra alla sezione~\ref{sec:note-volumi}, e $m_j$ è il numero di alberi carotati nella classe $j$.

Il volume complessivo degli alberi nella classe $j$ è dato da

\[ V_j = s\cdot\sum_{i=1}^{n_j} v_{ij} \]

dove $n_j$ è il numero totale di alberi campionati nella classe $j$, e $s$ è un fattore di scala come sopra alla sezione~\ref{sec:note-volumi}. (Si noti che $n_j \gg m_j$.)

Quindi l'incremento percentuale $pV$ di tutti gli alberi nell'area assestamentale è la media ponderata sui volumi delle classi diametriche:

\begin{equation}
    \label{eq:pV}
    pV = \frac{\sum_{j=1}^{M} pv_jV_j}{\sum_{j=1}^{M} V_j}
\end{equation}
dove $M$ è il numero di classi diametriche e $V_j$ è il volume totale della classe $j$.

La ponderazione per volume fa sì che gli alberi di maggiori dimensioni, che contribuiscono di più alla massa legnosa, pesino di più nella stima dell'incremento di gruppo.

Equivalentemente, a ogni albero campione $i$ non carotato possiamo attribuire l'incremento percentuale della sua classe diametrica $j$, $pv_i = pv_j$, e sostituire l'equazione~\ref{eq:pV} con un calcolo ponderato sui volumi dei singoli alberi:
\begin{equation}
    \label{eq:pV2}
    pV = \frac{\sum_{i=1}^{N}pv_iv_i}{\sum_{i=1}^{N}v_i}
\end{equation}

dove $N$ è il numero di tutti gli alberi campionati nell'area assestamentale.

L'incremento corrente $\Delta V$ (m$^3$) di un gruppo di piante è dato quindi dal volume complessivo del gruppo moltiplicato per l'incremento percentuale medio:

\[ \Delta V = V \cdot \frac{pV}{100} \]

Si noti che per ottenere i volumi totali dell'area assestamentale i volumi effettivamente campionati vengono scalati in proporzione alla superficie campionata, come sopra nella sezione~\ref{sec:note-volumi}.

% Group the sum in eq:pV2 by diameter class $j$. The numerator becomes (ignoring $s$, which cancels):
%
% $$
% \sum_j \left[\underbrace{\sum_{i \in \text{cored}} pv_iv_i}_{\text{actual } pv_i} + \underbrace{pv_j \sum_{i \in \text{non-cored}} v_i}_{\text{assigned } pv_j}\right]$$
%
% For equivalence with eq:pV we need this to equal $\sum_j pv_j \sum_{i=1}^{n_j} v_i$, i.e.:
%
%  $$pv_j \sum_{i \in \text{cored}} v_i = \sum_{i \in \text{cored}} pv_iv_i$$
%
% But this is the definition of $pv_j$ at line 164, rearranged:
%
%  $$pv_j = \frac{\sum_{i=1}^{m_j} pv_iv_i}{\sum_{i=1}^{m_j} v_i} \quad \Longrightarrow \quad pv_j \sum_{i \in \text{cored}} v_i = \sum_{i \in \text{cored}} pv_iv_i \quad$$


\subsection{Aggiornamento del programma dei tagli}
\label{sec:note:piano}

Il programma di tagli proposto nella relazione originale ha due difetti: per diverse particelle propone interventi colturali (seppure al di sotto dei limiti stabiliti) a intervalli minori di 10 anni, e non tiene conto degli incrementi nel valutare l'entità di interventi futuri.

Abbiamo cercato di affrontare entrambe le problematiche tramite una simulazione a tempo discreto ispirata al metodo matriciale di Usher~\cite{usher:1966}, comunemente usato nella modellizzazione delle dinamiche forestali. La simulazione opera su un orizzonte temporale definito (ad es.\ 2026--2040) con passo annuale, e ad ogni anno seleziona le particelle da sottoporre a taglio tenendo conto sia della crescita avvenuta sia dei vincoli normativi.

\subsubsection{Parametri della simulazione}

La simulazione richiede i seguenti parametri:
\begin{itemize}
\item un \emph{volume obiettivo} annuo $V_o$ (m$^3$), che rappresenta il prelievo complessivo desiderato dall'azienda in un dato anno, sempre che le condizioni di produttività del bosco lo permettano;
\item un \emph{intervallo minimo} $t$ (anni) tra due interventi successivi sulla stessa particella;
\item un tasso di \emph{mortalità} annuo $m$ (\%), applicato uniformemente a tutte le piante. In assenza di precisi dati stazionali, si adotta un tasso del 1\%~\cite{lacasia:2025}.
\end{itemize}

\subsubsection{Tabella di crescita}
Prima della simulazione, si costruisce una tabella di crescita a partire dagli incrementi osservati (sezione precedente). Per ogni combinazione di compresa, specie e classe diametrica, la tabella riporta l'incremento percentuale medio $pv$ e l'incremento corrente di diametro $\Delta d$. Quando per una data combinazione non sono disponibili dati diretti, si utilizza il valore della classe diametrica più vicina della stessa specie e compresa.

\subsubsection{Ciclo annuale}

Per ogni anno dell'orizzonte di simulazione:

\begin{enumerate}
    \item \textbf{Ordinamento.} Le particelle vengono ordinate per provvigione matura decrescente (m$^3$/ha), in modo da intervenire prioritariamente sui soprassuoli più densi, riducendone la competizione e permettendo ulteriore crescita in quelli meno densi.
    \item \textbf{Selezione e taglio.} Per ciascuna particella, nell'ordine stabilito:
    \begin{itemize}
        \item Se l'ultimo intervento è avvenuto meno di $t$ anni prima, la particella viene esclusa.
        \item Altrimenti, si simula un taglio colturale dal basso secondo le regole descritte nella sezione~\ref{sec:note-ripresa}, applicando i medesimi vincoli di volume e area basimetrica. Gli alberi prelevati vengono rimossi dalla simulazione.
        \item Il volume prelevato si accumula nel totale annuo; quando questo raggiunge $V_o$, non si interviene in ulteriori particelle nell'anno corrente.
    \end{itemize}

    \item \textbf{Crescita.} Al termine dei tagli, tutte le piante rimanenti crescono di un anno:
    \begin{itemize}
        \item Il volume di ciascun albero viene aggiornato per crescita e mortalità:
        \[v \leftarrow v\cdot(1 + pv/100)\cdot(1 - m/100) \]
        Il tasso di crescita $pv$ è specifico di ciascun albero; il tasso di mortalità $m$, proprio della popolazione, viene modellato quale riduzione uniforme del volume di ogni individuo.
        \item Il diametro viene incrementato di $\Delta d$, con conseguente eventuale passaggio alla classe diametrica superiore.
        \item L'età media di ogni particella viene incrementata di un anno. (Nella realtà probabilmente l'aumento è superiore perché i tagli privilegiano le piante più piccole.)
    \end{itemize}
\end{enumerate}

Il risultato della simulazione è un programma dei tagli che indica, per ogni anno e particella, il volume prima dell'intervento, il prelievo e il volume residuo.

\section{Curve ipsometriche e classi diametriche}
\label{sec:ipsometrie}
In questa sezione è riportata, per ciascuna compresa, la distribuzione grafica in classi diametriche del numero di piante per ettaro. Presentiamo un solo grafico di questo tipo per ciascuna compresa, in modo da evidenziare la partecipazione delle diverse specie. Sono esclusi i diametri (grossi) per i quali la frequenza di alberi è inferiore a 0,5 piante / ettaro.

Sono riportate inoltre, per ciascuna compresa e per ciascuna specie ivi prevalente, le curve ipsometriche complete delle relative equazioni interpolanti, come descritto nella sezione~\ref{sec:note:ipsometrie}.

Le equazioni sono calcolate con un modello di regressione logaritmico, solo per combinazioni di compresa e genere per le quali sono state misurate almeno 10 piante ($n \geq 10$).

\subsection{Serra}

La distribuzione in classi diametriche a Serra è la seguente. Il bosco è dominato dall'abete, ma sono presenti anche faggio, castagno, douglas, e pino nero.

\subsection*{Serra: Classi diametriche}
@@gcd(compresa=Serra,alberi=alberi.csv)

\subsection*{Serra: curve ipsometriche, veduta d'insieme}
Presentiamo le curve ipsometriche per le specie prevalenti nella compresa di Serra: abete, castagno, douglas, faggio, pino nero. Le rimanenti specie sono troppo poco frequenti per permettere un'analisi significativa delle altezze.

@@gci(compresa=Serra,alberi=altezze.csv,equazioni=equazioni_ipsometro.csv,genere=Abete,genere=Castagno,genere=Douglas,genere=Faggio,genere=Pino Nero)
\clearpage

\subsection*{Serra: ipsometria abete}
@@gci(compresa=Serra,genere=Abete,alberi=altezze.csv,equazioni=equazioni_ipsometro.csv)

\subsection*{Serra: ipsometria castagno}
@@gci(compresa=Serra,genere=Castagno,alberi=altezze.csv,equazioni=equazioni_ipsometro.csv)
\clearpage

\subsection*{Serra: ipsometria douglas}
@@gci(compresa=Serra,genere=Douglas,alberi=altezze.csv,equazioni=equazioni_ipsometro.csv)

\subsection*{Serra: ipsometria faggio}
@@gci(compresa=Serra,genere=Faggio,alberi=altezze.csv,equazioni=equazioni_ipsometro.csv)
\clearpage

\subsection*{Serra: ipsometria pino nero}
@@gci(compresa=Serra,genere=Pino Nero,alberi=altezze.csv,equazioni=equazioni_ipsometro.csv)
\clearpage

\subsection{Fabrizia}

La distribuzione in classi diametriche a Fabrizia è la seguente. Il bosco è dominato dal faggio, con una cospicua presenza di abete e pino nero.

\subsection*{Fabrizia: Classi diametriche}
@@gcd(compresa=Fabrizia,alberi=alberi.csv)

\subsection*{Fabrizia: curve ipsometriche, veduta d'insieme}
Le specie prevalenti, per le quali abbiamo potuto calcolare le equazioni interpolanti, sono abete, douglas, faggio e pino nero.
@@gci(compresa=Fabrizia,alberi=altezze.csv,equazioni=equazioni_ipsometro.csv,genere=Abete,genere=Douglas,genere=Faggio,genere=Pino Nero)
\clearpage

\subsection*{Fabrizia: ipsometria abete}
@@gci(compresa=Fabrizia,genere=Abete,alberi=altezze.csv,equazioni=equazioni_ipsometro.csv)

\subsection*{Fabrizia: ipsometria douglas}
@@gci(compresa=Fabrizia,genere=Douglas,alberi=altezze.csv,equazioni=equazioni_ipsometro.csv)
\clearpage

\subsection*{Fabrizia: ipsometria faggio}
@@gci(compresa=Fabrizia,genere=Faggio,alberi=altezze.csv,equazioni=equazioni_ipsometro.csv)

\subsection*{Fabrizia: ipsometria pino nero}
@@gci(compresa=Fabrizia,genere=Pino Nero,alberi=altezze.csv,equazioni=equazioni_ipsometro.csv)
\clearpage

\subsection{Capistrano}

La distribuzione in classi diametriche a Capistrano è la seguente. Il bosco è dominato da abeti e pini.
\subsection*{Capistrano: Classi diametriche}
@@gcd(compresa=Capistrano,alberi=alberi.csv)

\subsection*{Capistrano: curve ipsometriche, veduta d'insieme}
Le specie prevalenti, per le quali abbiamo potuto calcolare le equazioni interpolanti, sono abete, douglas, pino marittimo e pino nero.
@@gci(compresa=Capistrano,alberi=altezze.csv,equazioni=equazioni_ipsometro.csv)
\clearpage

\subsection*{Capistrano: ipsometria abete}
@@gci(compresa=Capistrano,genere=Abete,alberi=altezze.csv,equazioni=equazioni_ipsometro.csv)

\subsection*{Capistrano: ipsometria douglas}
@@gci(compresa=Capistrano,genere=Douglas,alberi=altezze.csv,equazioni=equazioni_ipsometro.csv)
\clearpage

\subsection*{Capistrano: ipsometria pino marittimo}
@@gci(compresa=Capistrano,genere=Pino Marittimo,alberi=altezze.csv,equazioni=equazioni_ipsometro.csv)

\subsection*{Capistrano: ipsometria pino nero}
@@gci(compresa=Capistrano,genere=Pino Nero,alberi=altezze.csv,equazioni=equazioni_ipsometro.csv)

\clearpage


\setlength{\LTleft}{0.1in}

\section{Stima dei volumi}
\label{sec:volumi}
Di seguito il risassunto dei volumi totali di ciascuna compresa, e poi le stime suddivise per singole particelle. I valori inferiori e superiori sono relativi all'intervallo fiduciario al 95\%.

@@tsv(alberi=alberi-calcolati.csv,per_compresa=si,per_particella=no,per_genere=no,intervallo_fiduciario=si,totali=si)

\subsection*{Stima dei volumi: Serra}
@@tsv(alberi=alberi-calcolati.csv,compresa=Serra,per_compresa=no,per_particella=si,per_genere=no,intervallo_fiduciario=si,totali=si)
\clearpage

\subsection*{Stima dei volumi: Fabrizia}
@@tsv(alberi=alberi-calcolati.csv,compresa=Fabrizia,per_compresa=no,per_particella=si,per_genere=no,intervallo_fiduciario=si,totali=si)

\subsection*{Stima dei volumi: Capistrano}
@@tsv(alberi=alberi-calcolati.csv,compresa=Capistrano,per_compresa=no,per_particella=si,per_genere=no,intervallo_fiduciario=si,totali=si)

\clearpage


\section{Ripresa attuale}
\label{sec:ripresa}
In base alle regole descritte nella sezione~\ref{sec:note-ripresa}, la ripresa totale per compresa è la seguente.

@@tpt(alberi=alberi-calcolati.csv,per_compresa=si,per_particella=no,col_prelievo=si,totali=si)

Di seguito, per ciascuna compresa, la ripresa per particella. Per ogni particella riportiamo anche il comparto e l'età media per permettere di verificare manualmente le percentuali massime di prelievo secondo i criteri di cui sopra.

\subsection*{Ripresa: Serra}
@@tpt(alberi=alberi-calcolati.csv,compresa=Serra,per_compresa=no,totali=si)

\subsection*{Ripresa: Fabrizia}
@@tpt(alberi=alberi-calcolati.csv,compresa=Fabrizia,per_compresa=no,totali=si)

\subsection*{Ripresa: Capistrano}
@@tpt(alberi=alberi-calcolati.csv,compresa=Capistrano,per_compresa=no,totali=si)

\clearpage


\section{Ceduo}
\label{sec:ceduo}
Come già anticipato, gli interventi nei cedui vengono effettuati a turni di 12, 15, o 18 anni, a seconda delle condizioni del sopprassuolo.

\begin{description}
    \item[Turno di 12 anni: avviamento a fustaia.] Nei cedui di castagno dove è presente la rinnovazione di conifere, si interviene a intervalli di 12 anni con rilascio di 50 matricine/ha per due turni, in modo da favorire la competizione verticale e l'avviamento a fustaia.

    \item[Turno di 15 anni: gestione ordinaria.] Dove le condizioni stazionali sono normali, si interviene a intervalli di 15 anni con rilascio di 50 matricine/ha per due turni, in attesa dell'insediamento della rinnovazione di conifere o di interventi di semina.

    \item[Turno di 18 anni: miglioramento stazionale.] Dove il bosco è più rado, si interviene a intervalli di 18 anni con rilascio di 75 matricine/ha per due turni, con l'obbiettivo di incrementare la copertura e la lettiera e migliorare la fertilità stazionale.
\end{description}

Di seguito il calendario degli interventi pianificati per i cedui:



\section{Piano dei tagli}
\label{sec:piano}
@@tpdt(alberi=alberi-calcolati.csv,calendario=calendario-tagli-2011-2025.csv,volume_obiettivo=20000,totali=si)


\clearpage
\section{Dettaglio per particella}
\label{sec:particelle}
\maybeinput{sec-particelle}

\printbibliography

\end{document}
