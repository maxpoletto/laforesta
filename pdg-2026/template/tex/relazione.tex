\documentclass[a4paper,11pt]{article}
\usepackage[utf8]{inputenc}
\usepackage[italian]{babel}
\usepackage{url}
\usepackage{graphicx}
\usepackage{geometry}
\usepackage{hyperref}
\usepackage{float}
\usepackage{longtable}
\usepackage{verbatim}
\geometry{margin=2cm}
\setlength{\LTleft}{1cm}

% Typography: serif body text
\usepackage{libertinus}
\usepackage[T1]{fontenc}
% Sans-serif section titles (Gill Sans-like)
\usepackage{gillius}
\usepackage{sectsty}
\allsectionsfont{\sffamily}
\setlength{\parindent}{0pt}
\usepackage[style=authoryear]{biblatex}
\addbibresource{ref.bib}

\newif\ifincludesections \includesectionsfalse
\newcommand{\maybeinput}[1]{\ifincludesections \input{#1} \fi}

\title{\sffamily Documentazione integrativa al Piano di Assestamento}
\date{\sffamily Febbraio 2026}
\author{\sffamily Società Agricola La Foresta\\\sffamily Serra San Bruno}

\begin{document}
\maketitle

\section{Introduziones}

Il presente documento contiene la documentazione integrativa al Piano di Assestamento della Società Agricola La Foresta, come richiesto dal Dipartimento Politiche della Montagna, Foreste, Forestazione e Difesa del Suolo della Regione Calabria con Protocollo 464707 del 25/06/2025.

La relazione si articola in 7 parti.
\begin{itemize}
\item La parte~\ref{sec:note-metodologia} chiarisce alcuni metodi usati per redigere il piano, in particolare per quanto riguarda il calcolo degli incrementi e l'elaborazione del programma di taglio.
\item La parte~\ref{sec:ipsometrie-diametri} risponde al punto 1 della richiesta della regione: illustra per ogni specie principale le relative curve ipsometriche con equazioni interpolanti e la distribuzione grafica in classi diametriche del numero di alberi per ettaro.
\item La parte~\ref{sec:volumi} risponde al punto 2 della richiesta: presenta la cubatura di tutti i soprassuoli usando le tavole prescritte dalle linee guida.
\item La parte~\ref{sec:ripresa} risponde al punto 3 della richiesta: quantifica la ripresa in relazione alla stima dei volumi.
\item La parte~\ref{sec:ceduo} risponde al punto 4 della richiesta: espone il piano di regolarizzazione dei cedui di castagno.
\item La parte~\ref{sec:piano} presenta un piano di tagli aggiornato, redatto tenendo conto degli incrementi secondo i criteri descritti nella parte~\ref{sec:note-metodologia}.
\item La parte~\ref{sec:per-particella}, infine, risponde ai punti 5 e 6 della richiesta. Per ogni particella, indica le necessità colturali dei soprassuoli, descrive gli interventi colturali da eseguire, e fornisce, per ogni specie e classe diametrica, il numero di alberi per ettaro, il volume e l'area basimetrica per ettaro, la distribuzione delle altezze, e gli incrementi percentuali e correnti. Espone inoltre i volumi e la ripresa totali, suddivisi per specie.
\end{itemize}

\section{Note metodologiche}
\label{sec:note-metodologia}

Di seguito sono riportati alcuni chiarimenti sui rilevamenti e i metodi di calcolo usati nella stesura di questa relazione.

\subsection{Campionamento}

Le 177 aree di saggio già descritte nella relazione iniziale, ciascuna della dimensione di $\frac{1}{8}$ ha, sono state soggette a cavallettamento totale, fornendo i diametri di circa 9,900 piante. Per un sottoinsieme casuale di 1880 piante è stata rilevata anche l'altezza tramite ipsometro laser. Per un ulteriore sottoinsieme di {\bf circa 200 piante} è stato misurato l'incremento periodico di raggio $L_{10}$, ovvero lo spessore dei 10 anelli legnosi periferici successivi.

\subsection{Calcolo delle curve ipsometriche}

Le curve ipsometriche sono state calcolate per regressione sulle altezze e i diametri delle 1880 piante misurate, per ogni combinazione di specie e compresa comprendente più di 10 alberi campione. Le equazioni interpolante hanno forma logaritmica, $h = a \ln d + b$, dove $d$ è il diametro in cm e $h$ è l'altezza in metri.

\subsection{Calcolo dei volumi}
\label{sec:note-volumi}

I volumi dei soprassuoli sono calcolati secondo la metodologia alle pp.20--27 di~\cite{tabacchi:2011}. Nello specifico:

\begin{itemize}
    \item Per ogni albero cavallettato, l'altezza è stimata usando le equazioni interpolanti di cui sopra.
    \item Per tutti gli alberi di una singola specie in ciascuna area di saggio, il volume del fusto e dei rami grossi viene calcolato usando un'equazione di previsione della forma $v = b_1 + b_2d^2 + b_3d$, dove i coefficienti $b_i$ provengono dal Tabacchi. Analogamente, la varianza della somma di questi volumi è calcolata usando le matrici di varianza e i coefficienti del Tabacchi.
    \item L'intervallo fiduciario della stima dei volumi dell'intera area di saggio è dato da $\sqrt{\sum_{s}{f_s^2}}$, dove $f_s$ è l'intervallo fiduciario di ciascuna specie $s$, perché le equazioni di previsione sono state calibrate sulla base di campioni indipendenti.
    \item Infine, il volume dell'intera particella viene calcolato scalando la somma delle superfici delle aree di saggio all'intera particella. Anche in questo caso l'intervallo fiduciario è prudenziale, e presume che tutte le superfici di una particella siano completamente correlate tra di loro.
\end{itemize}

Da notare che i valori elencati nella sezione~\ref{sec:volumi} includono le piante sottomisura, al di sotto dei 17,5 cm di diametro (classe diametrica inferiore a 20 cm). Per contro, naturalmente, i volumi usati per la ripresa escludono queste piante.

\subsection{Calcolo della ripresa}
\label{sec:note-ripresa}

Secondo le prescrizioni di massima della~\cite{calabria:2011}, artt. 48 e 49, il prelievo durante i tagli colturali nelle fustaie è limitato in funzione della massa disponibile e dell'area basimetrica di ciascuna particella.

In base alla classe economica del comparto, si stabilisce una provvigione minima. Per la nostra azienda, le classi economiche sono le seguenti:

\begin{center}
    \small
    \begin{tabular}{llr}
    \hline
                     &                                          & Provvigione \\
    Classe economica & Caratteristiche                          & minima (m³/ha)\\
    \hline
    A                & Fustaia di abete                         & 350 \\
    B                & Fustaia mista di conifere e latifoglie   & 350 \\
    C                & Fustaia di pino                          & 250 \\
    D                & Fustaia di pino nero e latifoglie        & 250 \\
    E                & Fustaia di faggio e conifere             & 350 \\
    \hline
    \end{tabular}
\end{center}

Il prelievo massimo volumetrico è quindi dato dalla seguente tabella:

\begin{center}
    \small
    \begin{tabular}{lr}
    \hline
    Eccedenza su provvigione minima         & Prelievo massimo \\
    \hline
    $\ge 80\%$                              & 25\% \\
    $\ge 60\%, < 80\%$                      & 20\% \\
    $\ge 40\%, < 60\%$                      & 15\% \\
    $\ge 20\%, < 40\%$                      & 10\% \\
    $< 20\%$                                &  0\% \\
    \hline
    \end{tabular}
\end{center}

Inoltre, per fustaie fino a 60 anni di età, il prelievo non può superare una determinata percentuale dell'area basimetrica, secondo la seguente tabella:

\begin{center}
    \small
    \begin{tabular}{lr}
    \hline
    Età media                   & Limite area basimetrica \\
    \hline
    $\ge 30$ anni, $< 60$ anni  & 20\% \\
    $<30$ anni                  & 15\% \\
    \hline
    \end{tabular}
\end{center}

  Per ottemperare a questi duplici requisiti, procediamo con una simulazione
  albero per albero, basata sulle piante campionate.
  Per ogni particella~$p$, di area $a$~(ha):

\begin{enumerate}
\item Calcoliamo il volume totale $v$ (m$^3$) delle piante mature (sezione~\ref{sec:note-volumi}) e l'area basimetrica totale $g$  (m$^2$) delle stesse.

\item In base alla classe economica di $p$ e alla provvigione $v/a$, ricaviamo dalle prime due tabelle sopra il prelievo massimo percentuale $pp_{\max}$, e poniamo il prelievo massimo assoluto $V_{\max} = v \cdot pp_{\max}/100$.

\item In base all'età media di $p$, ricaviamo dalla terza tabella il limite percentuale di area basimetrica $gp_{\max}$, e poniamo il limite assoluto $G_{\max} = g \cdot gp_{\max}/100$.

\item Calcoliamo il fattore di espansione $s = a / a_s$, dove $a_s$ è la superficie complessiva delle aree di saggio in~$p$.

\item Ordiniamo le piante mature campionate per classe diametrica crescente (dai 20~cm in su), in modo da simulare un taglio colturale dal basso volto a rafforzare le piante di maggiori dimensioni.

\item Scorriamo le piante nell'ordine stabilito, accumulando il volume e l'area basimetrica (scalati per $s$) di ciascun albero $b$. Per ogni albero $b$, se l'aggiunta di $b$ non farebbe superare né $V_{\max}$ né $G_{\max}$, lo includiamo nel prelievo; altrimenti la simulazione si arresta. Il volume accumulato al termine costituisce il prelievo massimo per la particella.
\end{enumerate}

\subsection{Calcolo degli incrementi}

Per ogni albero campione $i$ assumiamo un incremento periodico di raggio $L_{10}$ (spessore degli ultimi 10 anelli legnosi, in mm) {\bf in base al $L_{10}$ effettivamente misurato su un albero campione nella stessa area di saggio}. Calcoliamo l'incremento percentuale $pv_i$ secondo la formula di Pressler. Definendo l'incremento corrente di diametro in cm anno$^{-1}$

\[ \Delta d_i = \frac{2L_{10}}{100} \]

otteniamo

\[ pv_i = c \cdot \frac{100\Delta d_i}{d_i} = c \cdot \frac{2L_{10}}{d_i} \]

dove $d_i$ è il diametro (cm) e $c$ è il coefficiente di Pressler. Prudenzialmente poniamo $c = 2$ per tutta la superficie boscata, nonostante studi svolti sull'Appennino Calabrese suggeriscano valori più alti (\cite{marziliano:2011} riscontra coefficienti di Schneider $K$ nell'intervallo $415-715$; analiticamente $c = K / 200$).

Per ogni area assestamentale (particella o compresa), procediamo alla determinazione dell'incremento percentuale medio $pv_j$ della classe diametrica $j$ quale media ponderata sui volumi dei singoli alberi campione, come secondo~\cite{corona:2007}: 

\[ pv_j = \frac{\sum_{i=1}^{n_j} pv_{ij}v_{ij}}{\sum_{i=1}^{n_j} v_{ij}} \]

dove $pv_{ij}$ è l'incremento percentuale di volume dell'$i$-esimo albero campionato della classe diametrica $j$, $v_{ij}$ è il suo volume calcolato come sopra alla sezione~\ref{sec:note-volumi}, e $n_j$ è il numero di alberi campionati appartenente alla classe $j$.

Analogamente, l'incremento percentuale $pV$ di tutti gli alberi nell'area assestamentale è la media ponderata sui volumi delle classi diametriche:

\[ pV = \frac{\sum_{j=1}^{M} pv_jV_j}{\sum_{j=1}^{M} V_j} \]

dove $V_j$ è il volume totale della $j$-esima classe diametrica e $M$ è il numero di classi diametriche.

Algebraicamente ciò si riduce a

\[
    pV = \frac{\sum_{j=1}^{M}\left[\frac{\sum_{i=1}^{n_j} pv_{ij}v_{ij}}{\sum_{i=1}^{n_j}v_{ij}}V_j\right]}{\sum_{j=1}^{M}\sum_{i=1}^{n_j} v_{ij}}
       = \frac{\sum_{i=1}^{N}pv_iv_i}{\sum_{i=1}^{N}v_i}
\]

dove $N$ è il numero di tutti gli alberi $i$ campionati nell'area assestamentale.

(La ponderazione per volume fa sì che gli alberi di maggiori dimensioni, che contribuiscono di pìu alla massa legnosa, pesino di più nella stima dell'incremento di gruppo.)

L'incremento corrente $\Delta V$ (m$^3$) di un gruppo di piante è dato quindi dal volume complessivo del gruppo moltiplicato per l'incremento percentuale medio:

\[ \Delta V = V \cdot \frac{pV\cdot V}{100} \]

Si noti che per ottenere i volumi totali dell'area assestamentale i volumi effettivamente campionati vengono scalati in proporzione alla superficie campionata, come sopra nella sezione~\ref{sec:note-volumi}.

\subsection{Aggiornamento del programma dei tagli}

\cite{usher:1966}

\section{Curve ipsometriche e classi diametriche}
\label{sec:ipsometrie-diametri}
\maybeinput{sec-ipsometrie-diametri}

\section{Stima dei volumi}
\label{sec:volumi}
\maybeinput{sec-volumi}

\section{Ripresa attuale}
\label{sec:ripresa}
\maybeinput{sec-ripresa}

\section{Ceduo}
\label{sec:ceduo}
\maybeinput{sec-ceduo}

\section{Piano dei tagli}
\label{sec:piano}
\maybeinput{sec-piano}

\section{Dettaglio per particella}
\label{sec:per-particella}
\maybeinput{sec-per-particella}

\printbibliography

\end{document}
