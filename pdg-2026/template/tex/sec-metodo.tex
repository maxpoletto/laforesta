La Società Agricola La Foresta interviene da sempre nei propri boschi adottando una gestione forestale {\em closer-to-nature}. La gestione si basa sull'imitazione dei processi naturali, nei quali la foresta si rinnova attraverso la dinamica di piccoli gruppi o mediante la rimozione di singoli individui, mantenendo nel tempo una struttura disetanea e multi-generazionale. 

Nelle fustaie tale approccio prevede l'applicazione del diradamento continuo, ovvero il mantenimento di una copertura forestale permanente, salvo eventi meteorologici o fitopatologici eccezionali. Ad ogni intervento vengono effettuati prelievi selettivi finalizzati a favorire l'insediamento della rinnovazione naturale e a creare aperture di luce esclusivamente nelle aree in cui la rinnovazione risulta già stabilizzata. 

In presenza di ampie aperture causate da eventi meteorologici straordinari si procede, ove necessario, con interventi di semina. Nelle aree caratterizzate da sottobosco particolarmente competitivo, dovuto alla ridotta formazione di lettiera organica conseguente alla scarsa presenza di latifoglie, si interviene mediante piantumazione di specie autoctone. 

Per i cedui, in funzione della fertilità stazionale e della pendenza del terreno, viene adottato un turno di 15 o 18 anni. Nei cedui di castagno nei quali risulta affermata la rinnovazione di conifere di pregio (Abies alba, Pseudotsuga menziesii), si effettuano interventi con periodicità di circa 12 anni al fine di mantenere una competizione costante tra ceduo e conifere, favorendo la crescita in altezza di queste ultime e avviando progressivamente la conversione delle superfici cedue a fustaia. 
