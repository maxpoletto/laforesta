\documentclass[a4paper,11pt]{article}
\usepackage[utf8]{inputenc}
\usepackage[italian]{babel}
\usepackage{url}
\usepackage{graphicx}
\usepackage{geometry}
\usepackage{float}
\usepackage{longtable}
\geometry{margin=2cm}

% Typography: serif body text
\usepackage{libertinus}
\usepackage[T1]{fontenc}

% Sans-serif section titles (Gill Sans-like)
\usepackage{gillius}
\usepackage{sectsty}
\allsectionsfont{\sffamily}
\title{\sffamily Documentazione integrativa al Piano di Assestamento}
\date{\sffamily Gennaio 2026}
\author{\sffamily Società Agricola La Foresta\\\sffamily Serra San Bruno}
\begin{document}
\maketitle

\section{Introduzione}

Il presente documento contiene la documentazione integrativa al Piano di Assestamento della Società Agricola La Foresta, come richiesto dal Dipartimento Politiche Della Montagna Foreste Forestazione e Difesa del Suolo della Regione Calabria con Protocollo 464707 del 25/06/2025.

\section{Curve ipsometriche e classi diametriche}

In questa sezione è riportata, per ciascuna compresa, la distribuzione grafica in classi diametriche del numero di piante per ettaro. Presentiamo un solo grafico di questo tipo per ciascuna compresa, in modo da evidenziare la partecipazione delle diverse specie.

Sono riportate inoltre, per ciascuna compresa e per ciascuna specie ivi prevalente, le curve ipsometriche in forma grafica, complete delle relative equazioni interpolanti.

Le equazioni sono calcolate con un modello di regressione logaritmico, solo per combinazioni di compresa e genere per le quali sono state campionate almeno 10 piante ($n \geq 10$).

Sono esclusi i diametri (grossi) per i quali la frequenza di alberi è inferiore a 0,5 piante / ettaro.
\clearpage

\subsection{Serra}

La distribuzione in classi diametriche a Serra è la seguente. Il bosco è dominato dall'abete, ma sono presenti anche faggio, castagno, douglas, e pino nero.

\subsection*{Serra: Classi diametriche}
@@cd(compresa=Serra)
\clearpage
\subsection*{Serra: curve ipsometriche, veduta d'insieme}
@@ci(compresa=Serra)
\clearpage
\subsection*{Serra: ipsometria abete}
@@ci(compresa=Serra,genere=Abete)
\clearpage
\subsection*{Serra: ipsometria castagno}
@@ci(compresa=Serra,genere=Castagno)
\clearpage
\subsection*{Serra: ipsometria faggio}
@@ci(compresa=Serra,genere=Faggio)
\clearpage
\subsection*{Serra: ipsometria pino nero}
@@ci(compresa=Serra,genere=Pino Nero)
\clearpage

\subsection{Fabrizia}

La distribuzione in classi diametriche a Fabrizia è la seguente. Il bosco è dominato dal faggio, con una cospicua presenza di abete e pino nero.

\subsection*{Fabrizia: Classi diametriche}
@@cd(compresa=Fabrizia)
\clearpage
\subsection*{Fabrizia: curve ipsometriche, veduta d'insieme}
@@ci(compresa=Fabrizia)
\clearpage
\subsection*{Fabrizia: ipsometria abete}
@@ci(compresa=Fabrizia,genere=Abete)
\clearpage
\subsection*{Fabrizia: ipsometria faggio}
@@ci(compresa=Fabrizia,genere=Faggio)
\clearpage
\subsection*{Fabrizia: ipsometria pino nero}
@@ci(compresa=Fabrizia,genere=Pino Nero)
\clearpage

\subsection{Capistrano}

La distribuzione in classi diametriche a Capistrano è la seguente. Il bosco è dominato da abeti e pini.
\subsection*{Capistrano: Classi diametriche}
@@cd(compresa=Capistrano)
\clearpage
\subsection*{Capistrano: curve ipsometriche, veduta d'insieme}
@@ci(compresa=Capistrano)
\clearpage
\subsection*{Capistrano: ipsometria abete}
@@ci(compresa=Capistrano,genere=Abete)
\clearpage
\subsection*{Capistrano: ipsometria pino nero}
@@ci(compresa=Capistrano,genere=Pino Nero)
\clearpage
\subsection*{Capistrano: ipsometria pino marittimo}
@@ci(compresa=Capistrano,genere=Pino Marittimo)
\clearpage

\section{Stima dei volumi}

Di seguito la stima dei volumi di ciascuna compresa, calcolata secondo la metodologia alle pp.20--27 di {\em Stima del volume e della fitomassa delle principali specie forestali italiane}, Tabacchi {\em et al.} 2011, scaricabile all'indirizzo {\small \url{https://www.inventarioforestale.org/wp-content/uploads/2022/10/tavole_cubatura.pdf}}.

Oltre alla cubatura totale, riportiamo una stima del numero di piante, nonché l'intervallo fiduciario al 95\% della stima dei volumi, ottenuto sempre con la metodologia del Tabacchi. Il margine d'errore per il volume totale è calcolato come la somma dei margini d'errore di ciascuna specie: rappresenta quindi il caso peggiore, in cui le particelle sono completamente correlate tra loro.

@@tsv(per_compresa=no,per_particella=no,per_genere=no ,intervallo_fiduciario=si,totali=si,stime_totali=si)
@@tsv(per_compresa=si,per_particella=no,per_genere=no ,intervallo_fiduciario=si,totali=si,stime_totali=si)

\subsection*{Serra}
@@tsv(compresa=Serra,per_genere=no,intervallo_fiduciario=si,totali=si,stime_totali=si)
\clearpage
\subsection*{Stima dei volumi: Fabrizia}
@@tsv(compresa=Fabrizia,per_genere=no,intervallo_fiduciario=si,totali=si,stime_totali=si)
\clearpage
\subsection*{Stima dei volumi: Capistrano}
@@tsv(compresa=Capistrano,per_genere=no,intervallo_fiduciario=si,totali=si,stime_totali=si)

\end{document}
