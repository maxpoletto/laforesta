\documentclass[a4paper,11pt]{article}
\usepackage[utf8]{inputenc}
\usepackage[italian]{babel}
\usepackage{url}
\usepackage{graphicx}
\usepackage{geometry}
\usepackage{float}
\usepackage{longtable}
\geometry{margin=2cm}

% Typography: serif body text
\usepackage{libertinus}
\usepackage[T1]{fontenc}

% Sans-serif section titles (Gill Sans-like)
\usepackage{gillius}
\usepackage{sectsty}
\allsectionsfont{\sffamily}
\title{\sffamily Documentazione integrativa al Piano di Assestamento}
\date{\sffamily Gennaio 2026}
\author{\sffamily Società Agricola La Foresta\\\sffamily Serra San Bruno}
\begin{document}
\maketitle

\section*{Esperimenti}

Dati di Serra.

\subsection*{Serra: Classi diametriche delle 10K piante di Sabatino}
@@cd(compresa=Serra,alberi=alberi.csv)

\subsection*{Serra: Classi diametriche delle piante misurate con ipsometro}
@@cd(compresa=Serra,alberi=altezze.csv)

\subsection*{Serra: curve ipsometriche originali}
@@ci(compresa=Serra,alberi=alberi.csv,equazioni=equazioni_originali.csv)

\subsection*{Serra: curve ipsometriche di alberi calcolati}
@@ci(compresa=Serra,alberi=alberi-calcolati.csv,equazioni=equazioni_ipsometro.csv)

\subsection*{Serra: curve ipsometriche di piante misurate con ipsometro}
@@ci(compresa=Serra,alberi=altezze.csv,equazioni=equazioni_ipsometro.csv)

\end{document}
